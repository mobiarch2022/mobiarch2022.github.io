\documentclass{easychair}
\titlerunning{MobiArch 2022}
\authorrunning{preface}
\pagenumbering{roman}
\setcounter{page}{1}
\begin{document}

\section*{Preface}
This volume contains the papers presented at MobiArch 2022: Workshop on Mobility in the Evolving Internet Architecture 2022 held on October 21, 2022 in Sydney.



There were 14 submissions. Each submission was reviewed by at least 3, and on the average 4.8, program committee members. The committee decided to accept 10 papers. The program also includes 3 invited talks.



\textbf{Abstract}



According to Cisco Annual Internet Report, over 70 percent of the global population will have mobile connectivity by 2023, while machine-to-machine (M2M) communications will account for half of the global connections. In addition, the COVID-19 pandemic outbreak accelerated the need for reliable broadband Internet and mobility support at scale.Accordingly, the last two decades have witnessed the continuous evolution of mobile network architectures to support the unprecedented growth in mobile data traffic and quality of service demands. Higher-throughput and lower-latency 5G networks, Network Function Virtualization, Network Slicing, Mobile Edge Computing, Open RAN, and RAN disaggregation are only a few of the research trends that promise to deliver a more reliable, faster, and secure mobile Internet experience.ACM MobiArch 2022 will continue to serve as a forum for discussing recent research trends and future directions gathered around all aspects of mobile communication technologies and mobile Internet architectures, such as system and radio access challenges that have to be faced to implement reliable, optimized, and scalable mobile networked systems.



\textbf{Introduction}



ACM MobiArch has advanced to its 17th anniversary as one consolidated tradition for bringing together an important number of researchers and industry players working in different aspects of mobile communication technologies and mobile network architectures. The workshop will continue to serve as a forum for discussing recent research trends and future directions gathered around all aspects of mobile communication technologies and mobile Internet architectures, such as system and radio access challenges that have to be faced to implement reliable, optimized, and scalable mobile networked systems.



\textbf{Scope}



The workshop solicits original, previously unpublished submissions covering network architectural issues and system design challenges for mobility in the evolving mobile Internet.




\begin{itemize}
	
\item Key research topics include but are not limited to:
	
\item Effects of COVID-19 Pandemic on mobile network architecture
	
\item Mobile computing and networking architectures in 5G-and-beyond
	
\item Software-defined networking (SDN) and network function virtualization (NFV)
	
\item Mobile Cloud Computing (MCC) and Mobile Edge Computing (MEC)
	
\item mmWave and Terahertz wireless communications
	
\item Acoustic and visible-light communications
	
\item Connected autonomous vehicles
	
\item IoT and smart cities
	
\item Machine to Machine (M2M) communications
	
\item Location management and positioning
	
\item Field trials, deployment and evaluation of innovative mobile Internet architectures
	
\item Blockchain for mobile networks
	
\item Machine Learning and AI for mobile networks
	
\item Security and privacy for mobile networks
\end{itemize}


\textbf{Committee Members}



\textbf{Dr. Yipeng Zhou, TPC co-chair} (yipeng.zhou@mq.edu.au) is a senior lecturer in computer science with School of Computing of Faculty of Science and Engineering (FSE) at Macquarie University, Australia. He received his PhD degree from The Chinese University of Hong Kong in 2012 and was the recipient of Australia Research Council Discover Early Career Research Award (ARC DECRA) in 2018. His research interests lie in distributed/federated learning, privacy protection and networking. He has published more than 90 peer-reviewed papers including IEEE INFOCOM, IEEE/ACM ToN, IEEE JSAC, IEEE TPDS, IEEE TMC, etc. He served as Area Chairs of IEEE ICME 2021, 2022, Publicity Co-Chair of IEEE DSAA 2022, TPC members of IJCAI, ICDCS, ICC, GLOBECOM, etc., and the leading guest editor of Digital Communications Networks, etc.



\textbf{Dr. Lorenzo Bertizzolo, TPC co-chair} (lbertizzolo@apple.com) is a wireless systems engineer at Apple working on 5G cellular software for iOS. He earned his Ph.D. in Computer Engineering at Northeastern University, in 2021. As a researcher, he worked with AT{\&}T Labs Research, Facebook Connectivity Labs, and Google, network infrastructure team. His focus is 5G/6G networks, wireless networked systems, software-defined networking for wireless systems, MIMO communications, and Non Terrestrial Networks. He earned the "Electrical and Computer Engineering Outstanding Research Award" from Northeastern University in 2020 and the "WiNTECH Best Paper Award" in 2019.



\textbf{Prof. Stefano Secci, TPC co-chair} (stefano.secci@cnam.fr) is professor of networking at Cnam, Paris, France. He received his M.Sc. degree in communications engineering from the Politecnico di Milano, Italy, in 2005, and a dual Ph.D. degree in computer science and networks from the Politecnico di Milano and T\'el\'ecom Paris-Tech, France. He held postdoctoral positions at NTNU, Norway, and GMU, United States. He was an associate professor with Sorbonne University-UPMC, Paris, France, from 2010 to 2018. He has also covered positions at NTNU, George Mason University, Fastweb Italia, and Ecole Polytechnique de Montr\'eal. His current research resides at the edges between intelligent communication networks, distributed systems, mobile computing and embedded systems. He is the head of the Network and IoT Systems research group at Cnam.



\textbf{Steering Committee}




\begin{itemize}
	
\item Jon Crowcroft, University of Cambridge
	
\item Katherine Guo, Bell Labs Research
	
\item Stefano Secci, Cnam, Paris
	
\item Jianhua He, University of Essex
	
\item Michele Nogueira, Federal Univ. of Parana
	
\item Xiaoming Fu, University of Goettingen
\end{itemize}


\textbf{Acknowledgement}



The organizers would like to thank the program committee for their service, as well as the invited speakers for sharing their research and perspective.



 



~\bigskip


\noindent
\begin{minipage}[t]{.4\textwidth}
September 18, 2022
\end{minipage}%
\hfill
\begin{minipage}[t]{.4\textwidth}\flushright
Yipeng Zhou\\
Lorenzo Bertizzolo\\
Stefano Secci
\end{minipage}

\end{document}